\documentclass[10pt,a5paper]{article}

\usepackage[T1]{fontenc}
\usepackage{enumitem}
\usepackage{geometry}
\usepackage{multicol}
\usepackage{titlesec}
\usepackage{hyperref}
\geometry{
	a4paper,
	left=1cm,
	right=1cm,
	top=1cm,
	bottom=1cm
}

\titleformat{\section}
{\Large\vspace*{-0.25em}}{}{0em}{\scshape}[\titlerule]

\titleformat{\subsection}[runin]
{\large\vspace*{-0.25em}}{}{0.25em}{\scshape}

\renewcommand{\maketitle}[5]{
	\begin{center}
	\huge{\scshape{#1}} \\
	\vspace*{0.25em}
	\large{#2} \\
	\vspace*{0.25em}
	\textbf{Mobile:} {#3} \hspace*{0.5em}
	\textbf{Email:} {#4} \hspace*{0.5em}
	\url{#5}
	\end{center}
}

\begin{document}
\maketitle
{Vinodh B}
{Software Developer | C/C++ Programmer | Linux Entusiast}
{+1 (913) 326-4029}
{vinodh.b.27@gmail.com}
{www.linkedin.com/in/vinodh27}

\thispagestyle{empty}

\section{Technical Skills}
\begin{itemize}[leftmargin=3mm]
	\item{\textbf{Programming Languages: }C/C++, Python, Shell Scripting, Java, Rust}
	\item{\textbf{Operating Systems: }Linux and all Unix like Systems}
	\item{\textbf{Tools \& Technologies: }Yocto Project, RDK, Git, Vim, Networking, GDB, QEMU, Latex }
\end{itemize}

\section{Experience}
\subsection{Comcast Corporation}
\hspace*{0.25em}\textit{Development Engineer 1 -- Platform Security Team}
\hfill Dec 2021 - July 2023
\begin{itemize}[leftmargin=6mm]
	\setlength\itemsep{0em}
	\item{Proficient in Yocto Project for custom Linux distribution creation for RDK devices.}
	\item{Implemented Tab Completion feature for a custom Walledgarden shell, enhancing secure SSH access to CPE devices.}
	\item{Integrated Software Bill of Materials (SBOM) functionality in the build system, enhancing supply chain security by tracking all software components in the final product.}
	\item{Enabled Address Space Layout Randomization (ASLR) for multiple RDKB devices, reducing memory-related security vulnerabilities through position-independent code.}
	\item{Developed XPKI certificates for TLS communications within internal components.}
\end{itemize}

\subsection{Cognizant}
\hspace*{0.25em}\textit{Program Analyst Trainee (Internship)}
\hfill Aug 2021 - Nov 2021
\begin{itemize}[leftmargin=6mm]
	\setlength\itemsep{0em}
	\item{Worked on backend web development using Spring Boot framework in Java.}
	\item{Developed RESTful API's for a backend web application.}
\end{itemize}

\section{Education}
\begin{itemize}[leftmargin=3mm]
	\setlength\itemsep{0em}
\item{
	\textbf{University of Central Missouri}
	\hfill 2023 {-} Present \\
	Masters {-} Computer Science
}
\item {
	\textbf{Bharath Institute of Higher Education and Research}
	\hfill 2017 {-} 2021 \\
	B.Tech {-} Computer Science Engineering {-} 8.4 / 10.0(CGPA)
}
\end{itemize}

\section{Projects}
\subsection{Vshell}
\hfill (Source Code: \href{https://github.com/v1n0dh/vshell}{\underline {vshell}})
\begin{itemize}[leftmargin=6mm]
	\setlength\itemsep{0em}
	\item{Developed a custom shell inspired by a coding challenge from \href{https://codingchallenges.fyi/challenges/challenge-shell}{\underline {codingchallenges.fyi}}, resembling bash.}
	\item{Implemented essential shell features such as command execution, I/O redirection, and pipes.
}
	\item{Incorporated tab completion for files, enhancing user convenience and efficiency.}
	\item{Implemented support for shell variables, providing flexibility and customization.}
	\item{Integrated a command history feature, allowing users to navigate through previously executed commands using up and down arrow keys.}
\end{itemize}
\vspace*{-0.5em}

\subsection{Bencode Parser}
\hfill (Source Code: \href{https://github.com/v1n0dh/Bencode-Parser}{\underline {Bencode-Parser}})
\begin{itemize}[leftmargin=6mm]
	\setlength\itemsep{0em}
	\item{Bencode Parser allows converting Bencoded data(Used in torrent files) into human readable JSON format.}
	\item{Developed in C++ and used Jsoncpp library for handling with JSON objects.}
	\item{Created an API which can be used in other projects while working with Bencoded data.}
\end{itemize}
\vspace*{-0.5em}

\subsection{Wifi-Scan}
\hfill (Source Code: \href{https://github.com/v1n0dh/dotfiles/blob/master/.scripts/wifi-scan}{\underline {wifi-scan}})
\begin{itemize}[leftmargin=6mm]
	\setlength\itemsep{0em}
	\item{Developed a user-friendly wrapper script for wpa\_supplicant, streamlining the initiation of network connections.}
	\item{Created in Bash, the script prompts users for available networks and facilitates seamless connection initiation.}
	\item{Designed to be particularly useful in GUI-less Linux environments, enhancing accessibility and usability.}
\end{itemize}

\end{document}
